\documentclass[SDSUThesis.tex]{subfiles} 
\begin{document}

\newpage
\pagenumbering{arabic}
\setcounter{tocdepth}{2}

% see this for  guidance https://www.cs.purdue.edu/homes/dec/essay.dissertation.html
\section{INTRODUCTION}

Software is becoming a vital part of companies. In 2011 Marc Andreessen, co-founder of the venture capital firm Andreessen-Horowitz,
famously claimed, ``Software is Eating The World'' \cite{Andreessen2001}. His argument was for the ever increasing importance of software
in all organizations big and small regardless of the industry.  With this important declaration, the production of new software is
going to be critical.  Just as important will be the effective measurement of how this software is produced.  

The goal of this work is to provide a basis of what data should be collected by a software development organization, and how that data
should be used to formulate a single number representing the software development score of an organization. A framework to 
achieve the score will be presented.  The  score
is not to be comparative between organizations, but to rather form a historical baseline for a specific organization.  Also, a software
system to collect, calculate, and report the score will be presented.

\subsection{BACKGROUND}

Throughout the history of software development, many values have been collected and used to measure the effectiveness of software development. 
Many metrics have been used in the past and are still being used.  A metric can be defined as the science of indirect measurement.
The following are some examples of metrics that can be collected about source code:
\begin{itemize}
\item SLOC - The number of Source Lines of Code 
\item NOM - The Number of Methods per class
\item Complexity - A numerical measure of the code complexity, some common examples are McCabe \cite{McCabe1976} and Halstead \cite{Halstead1977}
\item Design - The amount of coupling and cohesion present in the software code
\item Source Code Analysis - Tools that determine whether code adheres to specified set of rules. Common examples are PMD and findbugs.
\end{itemize}

However, the metrics on source code only explain part of the software development lifecycle.  
Most of the effort has been placed on measuring just the computer code and not the entire organization.  Software
development involves more than just source code.


\subsection{PREVIOUS SOFTWARE SCORING ATTEMPTS }

Sextant is a visualization tool for Java source code \cite{Winter2013}.  Sextant provides a graphical representation of the information related 
to a software system.  The tool provides the capability to provide custom rules which are specific to the domain or application.  However,
Sextant only provides metrics and analysis of the software code.  It provides no information regarding the rest of the 
software development lifecycle.  Also, the primary output of Sextant is visual graphs.  While these graphs do provide
useful information, they do not provide a single number to determine the performance of the software.

\textit{Software process mining} \cite{Rubin2007} provides
an algorithm and information for measuring a software engineering process from
the Software Configuration Management (SCM). Software process mining does provide a method for tracking all 
documentation that can be placed in SCM, not just source code.  The technique creates process models 
to understand the process of developing software and code. It is less focused on the output and results,
but it is more focused on conformance to a process or understanding the process. Once again, a single number is not produced.

One example of scoring software development is presented by Jones \cite{Jones2012}. The methodology looks
for the presence of various techniques used in software engineering.  The methodology provides a score based upon the 
productivity and quality increase of the technique being evaluated.  A couple example techniques are: 
automated source code analysis and continuous integration.  The end result is a score in range $[-10,10]$. 
While the single number score is nice, it does not account for the entirety of the software development lifecycle.

Much work has been done to determine metrics for source code, in fact 
entire books have been written on the topic of software metrics \cite{Jones1996, Putnam2013}. Yet, 
organizations still struggle to measure the production of software.
Little work exists
for scoring the entire software development organization. 

Quality is tested in \cite{Miguel2014}

See also \cite{Buse2010}.

\subsubsection{SEMAT}


SEMAT is being dubbed as the ``new software engineering'' \cite{Jacobson2014}.
Three parts to the kernel:
\begin{itemize}
\item Measurement
\item Categorization
\item Competencies
\end{itemize}

The seven dimensions:
\begin{itemize}
\item Opportunity
\item Stakeholders
\item Requirements
\item Software Systems
\item Work
\item Team
\item Way of Working
\end{itemize}



\subsection{ORGANIZATION OF THE WORK}

The remainder of this dissertation is divided into 5 chapters.  The next chapter provides
an overview of software, software development lifecycles, software engineering, and software
development organizations.  Chapter 3 introduces what is meant by the term data-driven
software engineering. Chapter 4 then provides an explanation of the Master Performance 
Indicator (MPI).  It will present the essential elements for calculating the MPI, as well
as the formulas, frameworks, and data necessary to produce the MPI. Chapter 5 provides
a technological framework for generating and storing the current and historical MPI values.
Chapter 6 demostrates how MPI can be implemented in a software development portion of 
a large financial institution.  Chapter 7 concludes the dissertation with a summary
of the results and some possible future directions for further enhancements. 

\end{document}