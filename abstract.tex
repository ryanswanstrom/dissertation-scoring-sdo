\documentclass[SDSUThesis.tex]{subfiles} 
\begin{document}

\newpage
\doublespacing

\begin{center}
ABSTRACT\\
\addcontentsline{toc}{section}{ABSTRACT}

\yourtitle \\
\yourname \\
\number\year\\
\end{center}

\par
Nearly every large organization on Earth is involved in software development at some level.  Some organizations
specialize in software development while other organizations only participate in software development out of
necessity. In both cases, the performance of the software development matters.  Organizations collect
vast amounts of data relating to software development.  What do the organizations do with that data? 
That is the problem.  Many organizations fail to do anything meaningful with the data.  

Another problem is knowing what data to collect.  There are many options, but certain data is more important than
others.  What data should a software development organization collect?

This paper plans to answer that question and present a framework to gather the 
right information and provide a score for an organization that
produces software.  The score is not to be comparative between organizations, but to be comparative for a specific
organization over time.  

The primary goal of this work is to provide a general framework for what a software development organization
should measure and how to report on those measurements.  The focus is providing a single number
to represent the entire
organization and not just the development efforts.  That single number is considered the CRI score. The secondary goal of this work is to provide
a framework for storing the necessary data.

\end{document}