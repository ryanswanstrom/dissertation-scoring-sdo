\documentclass[SDSUThesis.tex]{subfiles} 
\begin{document}

% new section
\section{Case Study: Scoring an SDO of a Large Financial Institution}

Data has been collected from the software development processes of
an SDO within a large financial institution.
The data collection was from 2008 to present. This section will provide the analysis performed and the MPI scores of that data. The data is incomplete but that is realistic of most oganizations.  Currently, the data
consists of availability, quality, and schedule.  Satisfaction and Requirements are being obtained. An example of the initial analysis is included in \cref{app:case}.

%These are notes

%Application(app) - a software system or a collection of other apps, for example: Wystar is an application but it is a collection of other apps namely WyConnect, WyBatch ...
%Project - A body of work involving zero or more apps in preparation for a release, infrastructure upgrades can be a project but no specific app is involved
%Release - A collection of projects being put in production on a specified date


%The WHAT of SDLC - These are the steps of SDLC that need to be completed for each project.

%1. Identify the Work/Task/Project
%  1. Get Initial Idea
%  2. Obtain Details
%2. Estimate
%  1. Create an Estimate (What is included? What is the output? days/dollars/hours/reqs)
%  2. Obtain Approval 
%  3. Quit or Go Forward
%3. Write BRD
%  1. Identify the Requirements
%  2. Detail the Requirements
%4. Write FSD
%  1. Find System Integrations
%  2. Identify Functional Specs
%  3. Detail the Functional Specs
%5. Development of all the tasks in FSD
%  1. Identify the Coding Tasks
%  2. Write the Code/Develop the solution
%  3. Write the Unit Tests
%6. Test
%  1. Create Test Plans and Cases
%  2. Run Test Plans and Cases
%7. Deployment
%  1. Create Deployment Steps
%  2. Run Deployment Steps
%8. Maintenance
%  1. Capture Bugs
%  2. Survey Users
% end of notes

\end{document}