\documentclass[SDSUThesis.tex]{subfiles} 
\begin{document}

% new section
\section{CASE STUDY: SCORING AN SDO OF A LARGE FINANCIAL INSTITUTION}

Data has been collected from the software development processes of
an SDO within a large financial institution.
The data collection was from 2007 to January 2015. As would be expected
in any organization, not all 5 elements have data for the entire duration.
CRI can still work because the calculations will only include
the available elements.  This section will provide the analysis performed and the CRI scores of that data.  An example of the initial analysis is included in \cref{app:case}.

\subsection{WITH HISTORICAL DATA}
    This first case will demonstrate how to calculate the CRI for an
    organization when historical data is present.  
    \subsubsection{QUALITY}
        Luckily, the data for quality goes from October 2007 to January 2015
        and contains 985 observations.
        This is important because historical data can be used to create the
        baseline quality function.  All quality for the years 2007 through 2013
        will be used as historical data for the purposes of creating
        the baseline quality function.  
    \subsubsection{AVAILABILITY}
    \subsubsection{SATISFACTION}
    \subsubsection{SCHEDULE}
    \subsubsection{REQUIREMENTS}
    \subsubsection{OVERALL}
        All of the months do not include the same elements.  

An CRI calculation analyzing historical performance.

%%\subsection{WITHOUT HISTORICAL DATA}

%%An CRI calculation with default values for quality, schedule and requirements.

\end{document}