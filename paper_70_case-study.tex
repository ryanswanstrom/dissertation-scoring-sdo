\documentclass[SDSUThesis.tex]{subfiles} 
\begin{document}

% new section
\section{CASE STUDY: SCORING AN SDO OF A LARGE FINANCIAL INSTITUTION}

    Data has been collected from the software development processes of
    an SDO within a large financial institution\footnote{All the raw data 
    files are available at \cite{Swanstrom2015}}.
    The data collection was from 2007 to January 2015. Only 4 elements had available
    data, and not all elements had data available for the entire time period.  The
    elements to be used are: quality, availability, schedule, and requirements.
    CRI is still effective when not all data is available.
    The overall CRI score will then be a 
    weighted average of the available elements. 
    This section will serve as a guide to preparing the CRI score
    based upon the available data.  

    \subsection{WITH HISTORICAL DATA}
        After the data has been collected, the raw data can be stored in the SDLC
        if desired.  Then scores for each of the 4 elements can be calculated.
        \subsubsection{QUALITY}
            The first step to dealing with the quality data is a quick
            analysis of the data.  Table \ref{tab:quality_desc} provides
            some descriptive statistics for the quality data.
            
            
            \begin{longtable}{@{}l rr rrr}
                \toprule%
                 \centering%
                 {\bfseries Column}
                 & {\bfseries Min}
                 & {\bfseries Max}
                 & {\bfseries Median}
                 & {\bfseries Mean}
                 & {\bfseries Variance} \\
                
                \cmidrule[0.2pt](r{0.125em}){1-1}%
                \cmidrule[0.2pt](lr{0.125em}){2-2}%
                \cmidrule[0.2pt](l{0.125em}){3-3}%
                \cmidrule[0.2pt](l{0.125em}){4-4}%
                \cmidrule[0.2pt](l{0.125em}){5-5}%
                % \midrule
                \endhead
                
                Development Effort & 0 & 26937 & 300 & 1637 & 18383464 \\
                \myrowcolour%
                Testing Effort & NA & NA & NA & NA  & NA\\
                SIT Defects & 0 & 1106 & 1 & 45.86 & 24528 \\
                \myrowcolour%
                UAT Defects & 0 & 277 & 0 & 10.28 & 1306 \\
                PROD Defects & 0 & 1216 & 5 & 51.5 & 20311 \\
                
                \bottomrule
                
                \multicolumn{6}{c}{985 obs. from 23 Application IDs from 2007 to 2015} \\
                
                
                \caption{QUALITY DATA DESCRIPTIVE STATISTICS}
                \label{tab:quality_desc}
            \end{longtable}
            
            Notice, the data for quality goes from October 2007 to January 2015
            and contains 985 observations.
            This is important because historical data can be used to create the
            baseline quality function.  All quality for the years 2007 through 2013
            will be used as historical data for the purposes of creating
            the baseline quality function.  Once the historical data is separated,
            it results in 799 observations to be used for creating the baseline
            quality function.  
            
            
            
            
        \subsubsection{AVAILABILITY}
        \subsubsection{SCHEDULE}
        \subsubsection{REQUIREMENTS}
        \subsubsection{OVERALL}
            All of the months do not include the same elements.  
    
    An CRI calculation analyzing historical performance.

%%\subsection{WITHOUT HISTORICAL DATA}

%%An CRI calculation with default values for quality, schedule and requirements.

\end{document}