\documentclass[SDSUThesis.tex]{subfiles} 
\begin{document}

% new section
\section{Conclusions}

There are many metrics that can be used to evaluate a Software Development Organization (SDO). 
Knowing which metrics to use and what they all mean can be a daunting task.  MPI is a
proposed solution to the difficulty of measuring an SDO.

Upon completion, this work shall identify:
\begin{itemize}
    \item Define \textbf{What} characteristics should be measured for an SDO
    \item Define \textbf{How} to measure those characteristics
    \item Map the relationship or lack of relationship with a Balanced Scorecard
    \item Create a Framework to store the necessary data for MPI
    \item Outline a Process to generate the MPI score
    \item Provide an example MPI score with real data
\end{itemize}

An entire software development organization (SDO) needs to be measured and analyzed properly, not just the development portion. This document has provided an overview of what indicators need to be measured for an SDO, and how those indicators can be combined to form a single number indicating the performance score of a software development organization.  Also, a framework to store this data was discussed.

\end{document}